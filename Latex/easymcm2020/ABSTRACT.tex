\begin{abstract} 
\setlength{\parskip}{1em}
E-commerce is growing at an unprecedented rate all over the world. In the process of purchasing goods, ratings and reviews play a vital reference role. Companies are pursuing a comprehensive and understandable analysis of online market data to craft greater success.

In this paper, we seek to devise several approaches to analyze the product evaluation by exploring ratings, text-based reviews and other related indicators.

We first perform exploratory data analysis by generating the data quality report and studying the distribution of the most critical indicators. Then we preprocess reviews by a series of steps, including removing punctuations, converting abbreviations, etc. Besides, we extract the frequent features of each product using \textbf{WordCloud}.

We then build PRMP, a framework that defines the patterns, relationships, measures, and parameters within and between ratings and reviews. We use \textbf{SentiWordNet} to obtain the sentiment of a review and normalize star ratings and helpfulness. Through \textbf{association analysis}, we visualize the relationship using a set of heat maps and draw conclusions from them.  

After that, we propose a new approach to find a traceable measure for the product. We use \textbf{Entropy Weight Method} to obtain weights of the indicators. To identify reputation trends over time, we use the \textbf{ARIMA Model} to fit the reputation score. We select one of the hair dryer products as our main study object, calculate its score over time and give its most likely trending result.  We use the \textbf{Non-linear Programming Model} to find the best combination of text-based and rating-based measures to indicate potential success and failure. We apply the \textbf{Hovland Persuasion Model} to build a decision model that describes the indicators that influence customer decisions, achieving the combination of the theory of social psychology and real-life context. And to analyze specific quality descriptors, we classify words into eight categories using the \textbf{NRC Emotion Lexicon}. Then we match the emotional intensity with star ratings and find the relationship between them.

Finally, based on the established model and detailed analysis, we put forward practical suggestions for Sunshine Company's marketing plan to improve its product competitiveness.

	
	
	% 美赛论文中无需注明关键字。若您一定要使用,
	% 请将以下两行的注释号 '%' 去除,以使其生效
	% \vspace{5pt}
	% \textbf{Keywords}: MATLAB, mathematics, LaTeX.

\end{abstract}



